\documentclass{article}
\usepackage{amsmath}
\usepackage{amsfonts}
\usepackage{mathtools}
\usepackage[ruled]{algorithm2e}
\usepackage{graphicx}
\graphicspath{ {./images/} }

\title{Smoking study \\ 
    \large Studio e analisi dei dati di soggetti fumatori e non per predizione}
\author{Natasha Fabrizio Matricola: 717446 \\
Email: \textit{n.fabrizio@studenti.uniba.it} \\
        Francesco Saverio Cassano Matricola: 716133 \\
        Email: \textit{f.cassano45@studenti.uniba.it} \\
    Progetto di Ingegneria della Conoscenza}
    
\date{}

\begin{document}

    \maketitle

    \newpage

    \tableofcontents{}

    \newpage



\section{Introduzione}

Il sistema è in grado di prevedere se un soggetto è potenzialmente un fumatore o meno a seconda dei valori riscontrati nel dataset preso in considerazione. In più l'utilizzatore del programma, può inserire dei valori (anche parziali) per poter comprendere se sia un soggetto potenzialmente fumatore o meno, in caso affermativo potrà anche ricevere un suggerrimento su quali valori migliorare per non essere considerato più un fumatore.

\section{Seconda Sezione}

"Lorem ipsum dolor sit amet, consectetur \textbf{adipiscing} elit, sed do eiusmod tempor incididunt ut labore et dolore magna aliqua. Ut enim ad minim veniam, quis nostrud exercitation ullamco laboris nisi ut aliquip ex ea commodo consequat. Duis aute irure dolor in reprehenderit in voluptate velit esse cillum dolore eu fugiat nulla pariatur. Excepteur sint occaecat cupidatat non proident, sunt in culpa qui officia deserunt mollit anim id est laborum."

\subsubsection{Seconda Sezione punto 2}
\paragraph{Dataset}

\end{document}
